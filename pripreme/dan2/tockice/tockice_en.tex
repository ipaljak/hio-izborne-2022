%%%%%%%%%%%%%%%%%%%%%%%%%%%%%%%%%%%%%%%%%%%%%%%%%%%%%%%%%%%%%%%%%%%%%%
% Problem statement
\begin{statement}[
  problempoints=100,
  timelimit=1 second,
  memorylimit=1024 MiB,
]{Points}

\chapter

A well-known Croatian singer and entertainer Jole is playing a game with his sweetheart that has to do with portraits being erased, months being stolen, calling, calling, and of course drawing the last segment between $N$ points in the plane. In this problem, we consider a simpler version which still keeps all relevant geometric aspects.

There are $N$ points in the plane. The singer and his sweetheart take turns alternately.
On each turn, the current player draws a segment connecting two of the given points, such that the new segment doesn't intersect the interior of any previously drawn segment. Jole goes first.
The player who can't make a move loses. For a given set of points, determine who wins if both players play optimally.

%%%%%%%%%%%%%%%%%%%%%%%%%%%%%%%%%%%%%%%%%%%%%%%%%%%%%%%%%%%%%%%%%%%%%%
% Input
\subsection*{Input}
The first line contains the integer $N$ from the task description.

The next $N$ lines describe the points. 
The $i$-th of these lines contains two space-separated integers $1 \leq x_i, y_i \leq 10^6$, the coordinates of the $i$-th point.

No three given points are collinear, and all points are distinct.
%%%%%%%%%%%%%%%%%%%%%%%%%%%%%%%%%%%%%%%%%%%%%%%%%%%%%%%%%%%%%%%%%%%%%%
% Output
\subsection*{Output}
On the only line of the output, output \texttt{"JOLE"} if Jole wins, or \texttt{"ONA"}, otherwise.

%%%%%%%%%%%%%%%%%%%%%%%%%%%%%%%%%%%%%%%%%%%%%%%%%%%%%%%%%%%%%%%%%%%%%%
% Scoring
\subsection*{Scoring}
{\renewcommand{\arraystretch}{1.4}
  \setlength{\tabcolsep}{6pt}
  \begin{tabular}{ccl}
 Subtask & Score & Constraints \\ \midrule
  1 & ? & $1 \le N \le 7$\\
  2 & ? & $1 \le N \le 3\,00$\\
  3 & ? & $1 \le N \le 1\,000$\\
  4 & ? & $1 \le N \le 100\,000$\\
\end{tabular}}

TODO Paljak check that you indeed want these subtasks

%%%%%%%%%%%%%%%%%%%%%%%%%%%%%%%%%%%%%%%%%%%%%%%%%%%%%%%%%%%%%%%%%%%%%%
% Examples
\subsection*{Examples}
\begin{tabularx}{\textwidth}{X'X'X}
\end{tabularx}

\textbf{Clarification of the first example:}
TODO

\end{statement}

%%% Local Variables:
%%% mode: latex
%%% mode: flyspell
%%% ispell-local-dictionary: "croatian"
%%% TeX-master: "../hio.tex"
%%% End:
