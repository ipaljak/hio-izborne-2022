%%%%%%%%%%%%%%%%%%%%%%%%%%%%%%%%%%%%%%%%%%%%%%%%%%%%%%%%%%%%%%%%%%%%%%
% Problem statement
\begin{statement}[
  problempoints=100,
  timelimit=1 second,
  memorylimit=1024 MiB,
]{Točkice}

Ljetni kamp mladih informatičara već se godinama održava na otoku Krku.  Mladi
informatičari ono malo slobodnog vremena u pravilu provode kupajući se na
\textit{Dražici}, popularnoj pješčanoj plaži, a do tamo ih prate odrasle,
odgovorne osobe.

Alenka i Bara dvije su (ne)odgovorne osobe. Umjesto da paze na djecu, odlučile
su vrijeme kratiti igrajući se u pijesku. U jednom trenutku Alenka nacrta $N$
točaka, pritom pazivši da niti jedne tri nisu kolinearne i prozbori:
\begin{quote}
``Zaigrajmo igru \textit{točkica}. Poteze ćemo vući naizmjence, a ja ću krenuti
  prva. U svakom ćemo potezu nacrtati dužinu koja spaja neke dvije točke, ali
  tako da ta dužina ne siječe niti jednu od prethodno nacrtanih dužina. Tko
  napravi zadnji potez je pobjednik!.''
\end{quote}
Za dani raspored točaka, odredite tko će pobijediti u igri \textit{točkica} uz
pretpostavku da obje igračice igraju optimalno.

%%%%%%%%%%%%%%%%%%%%%%%%%%%%%%%%%%%%%%%%%%%%%%%%%%%%%%%%%%%%%%%%%%%%%%
% Input
\subsection*{Ulazni podaci}
U prvom je retku prirodan broj $N$ iz teksta zadatka.

U $i$-tom od sljedećih $N$ redaka su dva prirodna broja $x_i$, $y_i$ $(1 \leq
x_i, y_i \leq 10^6)$ koji predstavljaju koordinate $i$-te točke.

Niti jedne tri točke neće biti kolinearne, te će svake dvije točke biti različite.
%%%%%%%%%%%%%%%%%%%%%%%%%%%%%%%%%%%%%%%%%%%%%%%%%%%%%%%%%%%%%%%%%%%%%%
% Output
\subsection*{Izlazni podaci}
U jedini redak ispišite \texttt{Alenka} ako će u igri pobjediti Alenka,
odnosno \texttt{Bara} ako će u igri pobjediti Bara.

%%%%%%%%%%%%%%%%%%%%%%%%%%%%%%%%%%%%%%%%%%%%%%%%%%%%%%%%%%%%%%%%%%%%%%
% Scoring
\subsection*{Bodovanje}
{\renewcommand{\arraystretch}{1.4}
  \setlength{\tabcolsep}{6pt}
  \begin{tabular}{ccl}
 Podzadatak & Broj bodova & Ograničenja \\ \midrule
  1 & ? & $1 \le N \le 7$\\
  2 & ? & $1 \le N \le 3\,00$\\
  3 & ? & $1 \le N \le 1\,000$\\
  4 & ? & $1 \le N \le 100\,000$\\
\end{tabular}}

%%%%%%%%%%%%%%%%%%%%%%%%%%%%%%%%%%%%%%%%%%%%%%%%%%%%%%%%%%%%%%%%%%%%%%
% Examples
\subsection*{Probni primjeri}
\begin{tabularx}{\textwidth}{X'X'X}
\sampleinputs{test/tockice.dummy.in.1}{test/tockice.dummy.out.1} &
\sampleinputs{test/tockice.dummy.in.2}{test/tockice.dummy.out.2} &
\sampleinputs{test/tockice.dummy.in.3}{test/tockice.dummy.out.3}
\end{tabularx}

\end{statement}

%%% Local Variables:
%%% mode: latex
%%% mode: flyspell
%%% ispell-local-dictionary: "croatian"
%%% TeX-master: "../hio.tex"
%%% End:
