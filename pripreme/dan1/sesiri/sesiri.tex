%%%%%%%%%%%%%%%%%%%%%%%%%%%%%%%%%%%%%%%%%%%%%%%%%%%%%%%%%%%%%%%%%%%%%%
% Problem statement
\begin{statement}[
  problempoints=100,
  timelimit=1 sekunda,
  memorylimit=512 MiB,
]{Šeširi}

Srednjoeuropska informatička olimpijada (CEOI) ove se godina održava u Lijepoj
našoj. Natjecatelji su dodatno uzbuđeni zbog najpoznatije hrvatske tradicije na
informatičkim olimpijadama. Naravno, radi se o darivanju šeširića s
prepoznatljivim "kockastim" uzorkom.

Ipak, gospodin Malnar odlučio je stati na kraj još jednoj tradiciji. Ove će se
godine dijeliti jednobojni šeširići, obojeni u crvenu ili bijelu boju. Gospodin
Malnar nije donio ovu odluku tek tako, već je to dio pomno promišljenog plana.

Naime, Malnar zna da će se u nekom trenutku na istom mjestu naći $N$ mladih
informatičara te da će svaki od njih na glavi nositi crveni ili bijeli
šeširić.  Naravno, svaki će informatičar vidjeti sve šeširiće osim svojeg te
će, po običaju, svi istovremeno viknuti koje boje misle da je njhov šeširić.


%%%%%%%%%%%%%%%%%%%%%%%%%%%%%%%%%%%%%%%%%%%%%%%%%%%%%%%%%%%%%%%%%%%%%%
% Input
\subsection*{Ulazni podaci}
U prvom i jedinom retku je prirodan broj $N$, dimenzija šahovske ploče.

%%%%%%%%%%%%%%%%%%%%%%%%%%%%%%%%%%%%%%%%%%%%%%%%%%%%%%%%%%%%%%%%%%%%%%
% Output
\subsection*{Izlazni podaci}

U prvi redak ispišite prirodan broj $K$, najveći broj kraljica koji se može postaviti na ploču poštujući zadano pravilo.

U $i$-tom od sljedećih $K$ redaka ispišite prirodne brojeve $R_i$ i $S_i$, redak i stupac u koji je stavljena $i$-ta kraljica.

Ako postoji više načina za postaviti kraljice, ispišite bilo koji.

%%%%%%%%%%%%%%%%%%%%%%%%%%%%%%%%%%%%%%%%%%%%%%%%%%%%%%%%%%%%%%%%%%%%%%
% Scoring
\subsection*{Bodovanje}
{\renewcommand{\arraystretch}{1.4}
  \setlength{\tabcolsep}{6pt}
  \begin{tabular}{ccl}
 Podzadatak & Broj bodova & Ograničenja \\ \midrule
  1 & 6 & $1 \le N \le 16$\\
  2 & 11 & $1 \le N \le 64$\\
  3 & 28 & $1 \le N \le 256$\\
  4 & 55 & $1 \le N \le 1024$\\
\end{tabular}}

%%%%%%%%%%%%%%%%%%%%%%%%%%%%%%%%%%%%%%%%%%%%%%%%%%%%%%%%%%%%%%%%%%%%%%
% Examples
\subsection*{Probni primjeri}
\begin{tabularx}{\textwidth}{X'X'X}
\sampleinputs{test/kraljice.dummy.in.1}{test/kraljice.dummy.out.1} &
\sampleinputs{test/kraljice.dummy.in.2}{test/kraljice.dummy.out.2} &
\sampleinputs{test/kraljice.dummy.in.3}{test/kraljice.dummy.out.3}
\end{tabularx}

\pagebreak
\textbf{Pojašnjenje trećeg probnog primjera:} 
\begin{figure}[H]
\setchessboard{maxfield=c3,showmover=false,label=false,normalboard,marginright=false,marginbottom=false}
\chessboard[setblack={}]
\chessboard[setblack={Qc2}]
\chessboard[setblack={Qc2,Qa1}]
\chessboard[setblack={Qc2,Qa1,Qb2}]
\chessboard[setblack={Qc2,Qa1,Qb2,Qa3}]

\chessboard[setblack={Qc2,Qa1,Qb2,Qa3,Qc1}]
\chessboard[setblack={Qc2,Qa1,Qb2,Qa3,Qc1,Qb1}]
\chessboard[setblack={Qc2,Qa1,Qb2,Qa3,Qc1,Qb1,Qb3}]
\chessboard[setblack={Qc2,Qa1,Qb2,Qa3,Qc1,Qb1,Qb3,Qc3}]
\chessboard[setblack={Qc2,Qa1,Qb2,Qa3,Qc1,Qb1,Qb3,Qc3,Qa2}]
\end{figure}

% We're done

\end{statement}

%%% Local Variables:
%%% mode: latex
%%% mode: flyspell
%%% ispell-local-dictionary: "croatian"
%%% TeX-master: "../hio.tex"
%%% End:
