%%%%%%%%%%%%%%%%%%%%%%%%%%%%%%%%%%%%%%%%%%%%%%%%%%%%%%%%%%%%%%%%%%%%%%
% Problem statement
\begin{statement}[
  problempoints=100,
  timelimit=1 sekunda,
  memorylimit=512 MiB,
]{Kraljice}

Dana je $N \times N$ šahovska ploča. Odredite najveći broj kraljica koji se može postaviti na ploču, poštujući sljedeće pravilo:
\begin{quote}
    Kraljicu je dozvoljeno staviti na bilo koje \textbf{prazno} polje koje trenutno napada \textbf{paran} broj već postavljenih kraljica.
\end{quote}

Napomena: Jednom kad je postavljena na ploču, kraljica napada \textbf{sva} polja u istom retku, stupcu i na dijagonalama.

%%%%%%%%%%%%%%%%%%%%%%%%%%%%%%%%%%%%%%%%%%%%%%%%%%%%%%%%%%%%%%%%%%%%%%
% Input
\subsection*{Ulazni podaci}
U prvom i jedinom retku je prirodan broj $N$, dimenzija šahovske ploče.

%%%%%%%%%%%%%%%%%%%%%%%%%%%%%%%%%%%%%%%%%%%%%%%%%%%%%%%%%%%%%%%%%%%%%%
% Output
\subsection*{Izlazni podaci}

U prvi redak ispišite prirodan broj $K$, najveći broj kraljica koji se može postaviti na ploču poštujući zadano pravilo.

U $i$-tom od sljedećih $K$ redaka ispišite prirodne brojeve $R_i$ i $S_i$, redak i stupac u koji je stavljena $i$-ta kraljica.

Ako postoji više načina za postaviti kraljice, ispišite bilo koji.

%%%%%%%%%%%%%%%%%%%%%%%%%%%%%%%%%%%%%%%%%%%%%%%%%%%%%%%%%%%%%%%%%%%%%%
% Scoring
\subsection*{Bodovanje}
{\renewcommand{\arraystretch}{1.4}
  \setlength{\tabcolsep}{6pt}
  \begin{tabular}{ccl}
 Podzadatak & Broj bodova & Ograničenja \\ \midrule
  1 & TODO & $1 \le N \le TODO$\\
  2 & TODO & $1 \le N \le TODO$\\
  3 & TODO & $1 \le N \le TODO$\\
\end{tabular}}

%%%%%%%%%%%%%%%%%%%%%%%%%%%%%%%%%%%%%%%%%%%%%%%%%%%%%%%%%%%%%%%%%%%%%%
% Examples
\subsection*{Probni primjeri}
\begin{tabularx}{\textwidth}{X'X'X}
\sampleinputs{test/kraljice.dummy.in.1}{test/kraljice.dummy.out.1} &
\sampleinputs{test/kraljice.dummy.in.2}{test/kraljice.dummy.out.2} &
\sampleinputs{test/kraljice.dummy.in.3}{test/kraljice.dummy.out.3}
\end{tabularx}

\textbf{Pojašnjenje trećeg probnog primjera:} 
\begin{figure}[H]
\setchessboard{maxfield=c3,showmover=false,label=false,normalboard,marginright=false,marginbottom=false}
\chessboard[setblack={}]
\chessboard[setblack={Qc2}]
\chessboard[setblack={Qc2,Qa1}]
\chessboard[setblack={Qc2,Qa1,Qb2}]
\chessboard[setblack={Qc2,Qa1,Qb2,Qa3}]

\chessboard[setblack={Qc2,Qa1,Qb2,Qa3,Qc1}]
\chessboard[setblack={Qc2,Qa1,Qb2,Qa3,Qc1,Qb1}]
\chessboard[setblack={Qc2,Qa1,Qb2,Qa3,Qc1,Qb1,Qb3}]
\chessboard[setblack={Qc2,Qa1,Qb2,Qa3,Qc1,Qb1,Qb3,Qc3}]
\chessboard[setblack={Qc2,Qa1,Qb2,Qa3,Qc1,Qb1,Qb3,Qc3,Qa2}]
\end{figure}

% We're done

\end{statement}

%%% Local Variables:
%%% mode: latex
%%% mode: flyspell
%%% ispell-local-dictionary: "croatian"
%%% TeX-master: "../hio.tex"
%%% End:
