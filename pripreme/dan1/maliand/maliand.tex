%%%%%%%%%%%%%%%%%%%%%%%%%%%%%%%%%%%%%%%%%%%%%%%%%%%%%%%%%%%%%%%%%%%%%%
% Problem statement
\begin{statement}[
  problempoints=100,
  timelimit=2 sekunde,
  memorylimit=512 MiB,
]{Maliand}

Braća Maliand, Adrian i Vedran, slobodno vrijeme provode igrajući računalne
igre na obiteljskom računalu. Pritom su vrlo strastveni pa stvari nerijetko
izmaknu kontroli, što ima vrlo negativan utjecaj na zdravlje njihove majke.
Njihova majka je inače umjetnica, a umjetnica mora biti zdrava. Stoga je
odlučila uvesti raspored kojim će za svakog brata propisati koje dane u tjednu
smije provesti igrajući se na računalu.

Budući da su Maliandi umjetnička obitelj, odlučili su da se njihov tjedan
sastoji od $N$ uzastopnih dana.  Majka će dopustiti Adrianu da provede točno
$K$ dana tjedno igrajući igre, dok će Vedranu dopustiti da provede točno $L$
dana tjedno igrajući igre.  Rasporede će im uručiti već danas, a Adrian i
Vedran će ih se pridržavati od sutra.

Kako ne bi ispala prestroga prema svojim sinovima, odlučila im je dopustiti
da se počnu pridržavati rasporeda počevši od proizvoljnog dana napisanog na
rasporedu. Naravno, nakon toga se strogo moraju pridržavati rasporeda redom
kako piše, uz pretpostavku da se raspored periodički ponavlja u beskonačnost.

Primjerice, pretpostavimo da je $N=3$, $K=2$, te da je Adrian dobio raspored
$(1,0,1)$, gdje $1$ označava da se taj dan smije igrati na računalu, a $0$
označava da ne smije. Ako se Adrian odlući pridržavati rasporeda od drugog
napisanog dana, to znaći da se sutra neće igrati, pa će se iduća dva dana
igrati, pa se opet jedan dan neće igrati, \ldots

Gospođa Maliand je svjesna da će Adrian i Vedran biti najsretniji (i
najglasniji) one dane kada se obojica mogu igrati na računalu, te zaključuje
da će odabrati početak pridržavanja rasporeda tako da maksimiziraju broj
takvih dana.  S druge strane, ona će tada biti najmanje sretna, pa želi
napraviti takve rasporede da broj dana kada se obojica mogu igrati na
računalu bude najmanji mogući uz pretpostavku da će Adrian i Vedran početak
pridržavanja rasporeda odabrati tako da taj broj maksimiziraju.

Pomozite gospođi Maliand odrediti rasporede koji će minimizirati
broj dana u tjednu kada će se oba dječaka smjeti igrati na računalu.

%%%%%%%%%%%%%%%%%%%%%%%%%%%%%%%%%%%%%%%%%%%%%%%%%%%%%%%%%%%%%%%%%%%%%%
% Input
\subsection*{Ulazni podaci}
U prvom su retku brojevi $N$, $K$ i $L$ iz teksta zadatka $(0 \le K, L \le N)$.

%%%%%%%%%%%%%%%%%%%%%%%%%%%%%%%%%%%%%%%%%%%%%%%%%%%%%%%%%%%%%%%%%%%%%%
% Output
\subsection*{Izlazni podaci}

U prvi redak ispišite broj dana u tjednu kada će se oba dječaka smjeti igrati
na računalu ako gospođa Maliand optimalno odredi rasporede.

U drugi redak ispišite Adrianov raspored u obliku binarnog stringa duljine $N$
koji sadrži $K$ jedinica. Pritom, znamenka $1$ označava da se Adrian taj dan
smije igrati na računalu, dok znamenka $0$ označava suprotno.

U treći redak ispišite Vedranov raspored u obliku binarnog stringa duljine $N$
koji sadrži $L$ jedinica. Interpretacija ovog ispisa analogna je interpretaciji
Adrianovog rasporeda iz prethodnog odlomka.

Ako postoji više točnih rješenja, ispišite bilo koje.

%%%%%%%%%%%%%%%%%%%%%%%%%%%%%%%%%%%%%%%%%%%%%%%%%%%%%%%%%%%%%%%%%%%%%%
% Scoring
\subsection*{Bodovanje}
{\renewcommand{\arraystretch}{1.4}
  \setlength{\tabcolsep}{6pt}
  \begin{tabular}{ccl}
 Podzadatak & Broj bodova & Ograničenja \\ \midrule
  1 & 5 & $1 \le N \le 13$\\
  2 & 50 & $1 \le N \le 5\,000$\\
  3 & 45 & $1 \le N \le 500\,000$\\
\end{tabular}}

Ako vaš program na nekom testnom podatku ispiše točan prvi redak, ali ponudi
netočan raspored u drugm ili trećem retku, osvojit će $20\%$ bodova predviđenih
za taj test podatak. Broj bodova nekog podzadatka odgovara najmanjem broju
bodova koje je vaše rješenje ostvarilo na nekom od testnih podataka koji čine
taj podzadatak.

%%%%%%%%%%%%%%%%%%%%%%%%%%%%%%%%%%%%%%%%%%%%%%%%%%%%%%%%%%%%%%%%%%%%%%
% Examples
\subsection*{Probni primjeri}
\begin{tabularx}{\textwidth}{X'X'X}
\sampleinputs{test/maliand.dummy.in.1}{test/maliand.dummy.out.1} &
\sampleinputs{test/maliand.dummy.in.2}{test/maliand.dummy.out.2} &
\sampleinputs{test/maliand.dummy.in.3}{test/maliand.dummy.out.3}
\end{tabularx}

\textbf{Pojašnjenje prvog probnog primjera:} Ako se i Adrian i Vedran odluče
pridržavati rasporeda od prvog napisanog dana, tada će se obojica na računalu
igrati trećeg i petog dana (počevši od sutra). Može se pokazati da gospođa
Maliand ne može napraviti bolje rasporede.

\textbf{Pojašnjenje drugog probnog primjera:} Budući da se Vedran uopće ne
snije igrati na računalu, rješenje je $0$ neovisno o Adrianovom rasporedu i
njeogovoj odluci o početku pridržavanja rasporda.

\textbf{Pojašnjenje trećeg probnog promjera:} Ako se Adrian odluči pridržavati
rasporeda od prvog napisanog dana, a Vedran od četvrtog napisanog dana, tada će
se obojica na računalu igrati četvrtog, petog, osmog, devetog i desetg dana
(počevši od sutra). Može se pokazati da gospođa Maliand ne može napraviti bolje
rasporede.

% We're done

\end{statement}

%%% Local Variables:
%%% mode: latex
%%% mode: flyspell
%%% ispell-local-dictionary: "croatian"
%%% TeX-master: "../hio.tex"
%%% End:
