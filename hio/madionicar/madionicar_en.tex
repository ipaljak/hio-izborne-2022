%%%%%%%%%%%%%%%%%%%%%%%%%%%%%%%%%%%%%%%%%%%%%%%%%%%%%%%%%%%%%%%%%%%%%%
% Problem statement
\begin{statement}[
  problempoints=100,
  timelimit=5 seconds,
  memorylimit=512 MiB,
]{Mađioničar}

You might have heard that in his free time, Mr.\ Malnar does magic. 
His recent appearance in the famous TV show
\textit{Penn \& Teller: Fool Us} took the world by storm.
He introduced himself as
\textit{The Magical Mr.\ Malnar},
pulled off an incredible mentalist trick, 
and swept everyone off their feet.

He started off by calling up an eager volounteer from the audience and 
asking them to think of any string of their choice that consists of exactly $N$ letters.
He then proceeded to entertain the audience, occasionally glancing at the volounteer,
and at the end he declared: ``the longest sub-palindrome\footnote{
A \textit{palindrome} is a string that reads the same backward or forward.
A \textit{substring} of a string is a string made up from the $l$-th through the $r$-rh letter of that string, for some
$1 \leq l \leq r \leq N$. A \textit{sub-palindrome} is a substring which is also a palindrome.
} of your string has length $L$''.
After the volounteer confirmed this is indeed correct, the audience was stunned.

However, observant viewers and close friends of Mr.\ Malnar suspect this was not
mind reading, but a clever selection of words that, when combined with excellent
reading of facial expressions, gives away enough information to pull off the trick. 

While it seemed like Mr.\ Malnar was merely fooling around, from time to time
he would mention some interval of numbers $[l, r]$, where $1 \leq l \leq r \leq N$
and briefly glance at the volounteer. There are rumors saying he is able to determine
whether or not the substring of the volounteer's string that consists of
the $l$-th through the $r$-th letter (inclusive) is a palindrome, based on their facial
expression alone.

You need to write a program which will confirm that Mr.\ Malnar, if the rumors are true,
was able to gather enough information to determine the longest sub-palindrome of the
secret string choosen by the volounteer.

%%%%%%%%%%%%%%%%%%%%%%%%%%%%%%%%%%%%%%%%%%%%%%%%%%%%%%%%%%%%%%%%%%%%%%
% Input
\subsection*{Interaction}
This is an interactive task. Your program must communicate with a program made
by the organizers which simulates the behaviour of the volounteer
from the task description.

Before interaction, your program should read an integer $N$, the length of the
secret string, from the standard input task statement.

After that, your program can send query requests by writing to the standard
output. Each query must be printed in a separate line and have the
form ``\texttt{?} \textit{l r}'', where $1 \le l \le r \le N$ holds. After
each query has been written, your program should \textbf{flush} the output
and read the \textit{answer} from the standard input. The answer is a $1$ if
the substring $[l, r]$ is a palindrome, or $0$ if it's not.
\textbf{Your program can make at most $200\,000$ such queries.}

After your program has deduced the length of the longest sub-palindrome, it should write a line to
the standard output in the form ``\texttt{!} \textit{L}'', where
$L$ is the said length.
After that, your program should \textit{flush} the output once more
and gracefully terminate its execution.

\textbf{Note:} You can download the sample source code from the judging system
that performs the interaction correctly, including the output flush.

%%%%%%%%%%%%%%%%%%%%%%%%%%%%%%%%%%%%%%%%%%%%%%%%%%%%%%%%%%%%%%%%%%%%%%
% Scoring
\subsection*{Scoring}

{\renewcommand{\arraystretch}{1.4}
  \setlength{\tabcolsep}{6pt}
  \begin{tabular}{ccl}
   Subtask & Score & Constraints \\ \midrule
    1 & 13 & $1 \leq N \leq 7\,500$ \\
    2 & 25 & $1 \leq N \leq 65\,000$ \\
    3 & 25 & $1 \leq N \leq 100\,000$, the secret string consists of letters \texttt{a} and \texttt{b} only \\
    4 & 37 & $1 \leq N \leq 100\,000$ \\
\end{tabular}}

%%%%%%%%%%%%%%%%%%%%%%%%%%%%%%%%%%%%%%%%%%%%%%%%%%%%%%%%%%%%%%%%%%%%%%
% Examples
\subsection*{Interaction Example}
{\renewcommand{\arraystretch}{1.4}
  \setlength{\tabcolsep}{6pt}
  \begin{tabular}{lcl}
    Output & Input & Comment \\ \midrule
      & 5 & The secret string has length $5$. In this example, the volounteer chose the string \texttt{neven}. \\
    \texttt{\frenchspacing? 1 1} & \texttt{1} & Substring \texttt{n} is a palindrome. \\
    \texttt{\frenchspacing? 2 3} & \texttt{0} & Substring \texttt{ev} isn't a palindrome. \\
    \texttt{\frenchspacing? 2 4} & \texttt{1} & Substring \texttt{eve} is a palindrome. \\
    \texttt{\frenchspacing? 3 5} & \texttt{0} & Substring \texttt{ven} isn't a palindrome. \\
    \texttt{\frenchspacing? 1 5} & \texttt{1} & Substring \texttt{neven} is a palindrome. \\
    \texttt{\frenchspacing! 5} & & Correct, the longest sub-palindrome is the whole string \texttt{neven}. \\
\end{tabular}}


%%%%%%%%%%%%%%%%%%%%%%%%%%%%%%%%%%%%%%%%%%%%%%%%%%%%%%%%%%%%%%%%%%%%%%
% We're done
\end{statement}

%%% Local Variables:
%%% mode: latex
%%% mode: flyspell
%%% ispell-local-dictionary: "croatian"
%%% TeX-master: "../hio.tex"
%%% End:
