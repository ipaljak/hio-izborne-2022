%%%%%%%%%%%%%%%%%%%%%%%%%%%%%%%%%%%%%%%%%%%%%%%%%%%%%%%%%%%%%%%%%%%%%%
% Problem statement
\begin{statement}[
  problempoints=100,
  timelimit=1 sekunda,
  memorylimit=512 MiB,
]{Mađioničar}

\textit{Čujte i počujte! Prvi put u vašem gradu i to samo večeras! Jedan, jedini i
neponovljivi -- Magični Malnar. Uživajte u mega popularnim točkama poput
iščeznuća janjetine, rekordno brzog guljenja tvrdo kuhanih jaja i određivanja
  najduljeg palindromskog podstringa pomoću mentalizma.}

Da, dobro ste čuli, Gospodin Malnar se u \textit{fušu} bavi magijom, a svjetsku
je slavu stekao nakon nastupa na popularnom televizijskom showu
\textit{Penn \& Teller: Fool Us}. U svom se nastupu svijetu predstavio kao
\textit{The Magical Mr. Malnar}, izveo nevjerojatno uvjerljivu mentalističku
točku i oduševio publiku.

Gospodin Malnar je nastup započeo nasumičnim odabirom volontera iz publike kojeg
je zamolio da po vlastitom izboru zamisli neku riječ koja se sastoji od $N$
malih slova engleske abecede. Zatim je neko vrijeme zabavljao publiku, ponekad
je bacio oko na volontera, a na kraju izjavio: ``najdulji palindromski podstring
tvoje zamišljene riječi je $L$''. Nakon što je volonter potvrdio istinitost ove
tvrdnje, publika je pala u trans.

Međutim, pažljiviji gledatelji i bliski prijatelji gospodina Malnara slute da
se ipak nije radilo o čitanju misli, već o pomno odabranim rečenicama koje, u
kombinaciji sa vrlo vještim čitanjem izraza lica, odaju dovoljnu količinu
informacija potrebnih za uspješan ishod točke.

Naime, gospodin Malnar je (naizgled zabavljajući publiku) s vremena na vrijeme
spomenuo neki interval brojeva $[l, r]$, gdje je $1 \le l \le r \le N$, a
nakon toga je nakratko bacio oko na volontera. Šuška se da je sposoban na
temelju izraza lica volontera odrediti je li podstring od $l$-tog do $r$-tog
slova (uključivo) njegove zamišljene riječi palindrom.

Možete li napisati program koji će potvrditi da je gospodin Malnar, ako su
nagađanja istinita, mogao prikupiti dovoljno informacija da odredi najdulji
palindromski podstring riječi koju je zamislio volonter.

%%%%%%%%%%%%%%%%%%%%%%%%%%%%%%%%%%%%%%%%%%%%%%%%%%%%%%%%%%%%%%%%%%%%%%
% Input
\subsection*{Interakcija}
Ovo je interaktivni zadatak. Vaš program treba uspostaviti dijalog sa programom
izrađenim od strane organizatora koji simulira ponašanje volontera iz teksta
zadatka.

Prije interakcije vaš program treba sa standardnog ulaza pročitati
prirodan broj $N$ koji predstavlja duljinu riječi koju je volonter
zamislio.

Nakon toga, vaš program može slati upite pisanjem na standardni izlaz. Svaki
upit treba biti ispisan u zaseban redak te poprimati oblik
``\texttt{?} \textit{l r}''
gdje vrijedi $1 \le l \le r \le N$. Nakon svakog ispisanog upita, vaš program
treba napraviti \textit{flush} izlaza te sa standardnog ulaza pročitati
\textit{odgovor} na upit -- broj $1$ ako je interval zamišljene riječi koji
započinje $l$-tim, a završava $r$-tim znakom palindrom, odnosno $0$ ako to
nije. Vaš program smije poslati najviše $??$ ovakvih upita.

Kada je vaš program odgonetnuo duljinu najduljeg palindromskog podstringa
zamišljene riječi, treba na standardni izlaz ispisati redak oblika
``\texttt{!} \textit{L}'' gdje $L$ predstavlja spomenutu duljinu. Nakon toga,
vaš program ponovno treba napraviti \textit{flush} izlaza i završiti
izvođenje.

\textbf{Napomena:} Putem sustava za evaluaciju možete preuzeti primjere
izvornih kodova koji na ispravan način obavljaju interakciju, uključujući
\textit{flush} izlaza.

%%%%%%%%%%%%%%%%%%%%%%%%%%%%%%%%%%%%%%%%%%%%%%%%%%%%%%%%%%%%%%%%%%%%%%
% Scoring
\subsection*{Bodovanje}

{\renewcommand{\arraystretch}{1.4}
  \setlength{\tabcolsep}{6pt}
  \begin{tabular}{ccl}
   Podzadatak & Broj bodova & Ograničenja \\ \midrule
    1 & ?? & ?? \\
    2 & ?? & ?? \\
    3 & ?? & ??
\end{tabular}}

%%%%%%%%%%%%%%%%%%%%%%%%%%%%%%%%%%%%%%%%%%%%%%%%%%%%%%%%%%%%%%%%%%%%%%
% Examples
\subsection*{Primjer interakcije}
{\renewcommand{\arraystretch}{1.4}
  \setlength{\tabcolsep}{6pt}
  \begin{tabular}{lcl}
    Izlaz & Ulaz & Napomena \\ \midrule
      & 5 & Zamišljena riječ sastoji se od $5$ slova. U ovom primjeru volonter je zamislio rijeć \texttt{"neven"} \\
    \texttt{\frenchspacing? 1 1} & \texttt{1} & Podstring \texttt{n} je palindrom. \\
    \texttt{\frenchspacing? 2 3} & \texttt{0} & Podstring \texttt{ev} nije palindrom. \\
    \texttt{\frenchspacing? 2 4} & \texttt{1} & Podstring \texttt{eve} je palindrom. \\
    \texttt{\frenchspacing? 1 5} & \texttt{1} & Podstring \texttt{neven} je palindrom. \\
    \texttt{\frenchspacing! 5} & & Točno, najdulji palindromski podstring je sama zamišljena riječ \texttt{"neven"} \\
\end{tabular}}


%%%%%%%%%%%%%%%%%%%%%%%%%%%%%%%%%%%%%%%%%%%%%%%%%%%%%%%%%%%%%%%%%%%%%%
% We're done
\end{statement}

%%% Local Variables:
%%% mode: latex
%%% mode: flyspell
%%% ispell-local-dictionary: "croatian"
%%% TeX-master: "../hio.tex"
%%% End:
