\subsection*{Zadatak Mađioničar}
\textsf{Pripremila: Paula Vidas}\\
\textsf{Potrebno znanje: palindromi}

Prvi podzadatak možemo riješiti binarnim pretraživanjem. Primijetimo da, ako je
neka riječ palindrom, onda je i riječ dobivena micanjem prvog i zadnjeg slova također palindrom.
To znači da, ako postoji postoji palindromski podstring duljine $l > 1$, onda postoji i palindromski podstring duljine $l - 2$ (ali ne nužno i duljine $l - 1$!).
Stoga, možemo binarnim pretraživanjem odvojeno naći najdulji parni i najdulji neparni palindromski podstring, te je odgovor veći od ta dva.
Za to nam treba manje od $2 \cdot \lceil{\log(n/2)}\rceil \cdot n$ upita.

Drugi podzadatak podzadatak riješit ćemo koristeći najviše $3n$ upita.
Za svaki $r$ ($1 \leq r \leq n$), naći ćemo najmanji $l$ za koji je
podstring $[l, r]$ palindrom (drugim riječima, $[l, r]$ je 
najdulji palindromski podstring koji završava $r$-tim slovom).
Na početku stavimo $l = 1$, te idemo redom $r = 1$, $r = 2$, ..., $r = n$.
Kada prelazimo s $r - 1$ na $r$, primijetimo da novi $l$ može biti najmanje $l - 1$ (jer micanjem prvog i zadnjeg slova palindrom ostaje palindrom). Dakle, pri prijelazu
prvo provjerimo je li podstring $[l - 1, r]$ palindrom. Ako je, smanjujemo $l$ za 1 te smo našli željeni $l$. Inače, sve dok podstring $[l, r]$ nije palindrom, povećavamo $l$ za 1.
Primijetimo da je uvijek $1 \leq l \leq n$ te se tijekom algoritma $l$ može smanjiti za 1 najviše $n$ puta,
pa se stoga može povećati za 1 najviše $2n$ puta, što daje najviše $3n$ upita.

U trećem podzadatku, kada se riječ sastoji od najviše dva različita slova, možemo 
s $n - 1$ upita otkriti riječ, tako što postavimo upit za svaki podstring duljine dva.
Ako je palindrom, onda su ta dva slova jednaka, inače su različita. Nakon što znamo riječ,
najdulji palindromski podstring možemo naći primjerice pomoću binarnog pretraživanja i \textit{hashiranja} ili koristeći Manacherov algoritam.

Četvrti podzadatak rješavamo koristeći najviše $2n$ upita.
Odvojeno ćemo naći najdulji neparni i najdulji parni palindromski podstring.
Riješimo prvo neparni slučaj. Neka je na početku $d = 1$, to će nam predstavljati duljinu
najduljeg do sad nađenog neparnog palindromskog podstringa. Idemo redom $i = 1$, $i = 2$, ..., $i = n$.
Za svaki $i$ radimo sljedeće: sve dok je podstring duljine $d + 2$ sa ``središtem'' u $i$
(to je podstring $[i - (d + 1) / 2, i + (d + 1) / 2]$) palindrom, povećavamo $d$ za 2.
Naravno, ako bi taj podstring izlazio izvan granica riječi, prestajemo i prelazimo na sljedeći $i$.
Analizirajmo sada broj upita. Postavili smo $(d - 1) / 2$ potvrdnih upita (kada je podstring bio palindrom) te najviše $n - (d + 1) / 2$ negativnih upita
(za svaki $i \leq n - (d + 1) / 2$ najviše jedan, a za veće $i$ nismo pitali nijedan upit jer je podstring izlazio izvan granica). To ukupno daje najviše $n - 1$ upit.
Parni slučaj rješavamo analogno. 
