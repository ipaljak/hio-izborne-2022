%%%%%%%%%%%%%%%%%%%%%%%%%%%%%%%%%%%%%%%%%%%%%%%%%%%%%%%%%%%%%%%%%%%%%%
% Problem statement
\begin{statement}[
  problempoints=100,
  timelimit=1 sekunda,
  memorylimit=512 MiB,
]{Mensza}

Gospodin Malnar nedavno je osnovao Menszu -- najveću, najprestižniju i jedinu
međunarodnu udrugu visokointeligentnih ljubitelja gastronomije. Pogađate, članom
udruge ne može postati bilo tko, već samo kandidati koji postignu izvrsne
rezultate na prijamnom ispitu. Dakako, ispit se sastoji od zahtjevnog IQ dijela
i vrlo zahtjevnog gurmanskog \sout{di}jela. Primjer tipičnog IQ dijela ispita
predstavit ćemo vam u ovom zadatku, dok ćete primjer gurmanskog jela imati
prilike okusiti tek nakon natjecanja.

Prijamnom ispitu pristupili su Alojzije, Benjamin i Cecilija. Gospodin Malnar
ih je odmjerio, brzo smislio pravedan zadatak, te se obratio Alojziju:

``Alojzije, najprije ću tebe pozvati u svoj ured i pokazat ću ti prirodan
  broj $A$.  Zatim ćeš mi ti na papir napisati niz brojeva $a = (a_1, a_2,
\ldots, a_{l_a})$, pri čemu njegova duljina $l_a$ ne smije biti veća od $L$''

Nakon što se uvjerio da je Alojzije shvatio svoj dio posla, obratio se Benjaminu:

``Benjamine, tebe ću sljedećeg pozvati u ured i pokazat ću ti prirodan
  broj $B \ne A$.  Zatim ćeš mi ti na papir napisati niz brojeva $b = (b_1, b_2,
\ldots, b_{l_b})$, pri čemu njegova duljina $l_b$ ne smije biti veća od $L$''

Potom se obratio i Ceciliji:

``Cecilijo, tebe ću posljednju pozvati u ured i pokazat ću ti niz $c$ koji
sam odredio na temelju nizova $a$ i $b$. Precznije, za svaki broj koji se barem
jednom pojavljuje u nizovima $a$ i $b$, u niz $c$ ću dodati koliko se ukupno puta
pojavljuje u uniji nizova $a$ i $b$. Također, elemente niza $c$ ću ti pokazati u
neopadajućem poretku. Primjerice, ako je $a = (1, 2, 4)$ i $b = (1, 1, 2, 3)$,
pokazat ću ti $c = (1, 1, 2, 3)$ zato što se brojevi $3$ i $4$ pojavljuju jednom,
broj $2$ dva puta i broj $1$ tri puta. Nakon što ti pokažem niz $c$, ti mi trebaš
odgovoriti koji je od brojeva $A$ i $B$ veći.''

Konačno, kada su se dojmovi malo slegli, obratio se i svim kandidatima zajedno:

``Imate 60 minuta da smislite strategiju pa krećemo s ispitom. Nakon toga
više ne smijete komunicirati. Ovaj postupak ćemo ponoviti par puta
\sout{da se uvjerim da se ne radi o sreći} dok ne stigne klopa.''

Vaš je zadatak osmisliti strategiju koja bi omogučila Alojziju, Benjaminu i
Ceciliji da prođu IQ dio ispita.

%%%%%%%%%%%%%%%%%%%%%%%%%%%%%%%%%%%%%%%%%%%%%%%%%%%%%%%%%%%%%%%%%%%%%%
% Input
\subsection*{Ulazni podaci}
U prvom je retku prirodan broj $L$ iz teksta zadatka.

U drugom je retku prirodan broj $Q$, broj scenarija koje morate obraditi. Svaki
scenarij odgovara nekoj interakciji koja se zbiva u uredu gospodina Malnara.

U $i$-tom od sljedećih $Q$ redaka opisan je $i$-ti scenarij. Redak će započeti
riječju \texttt{alojzije}, \texttt{benjamin} ili \texttt{cecilija}, ovisno o
tome kojeg je kandidata gospodin Malnar pozvao u ured.

Ako $i$-ti redak započinje riječju \texttt{alojzije}, tada se u nastavku retka
nalazi prirodan broj $A$.

Ako $i$-ti redak započinje riječju \texttt{benjamin}, tada se u nastavku retka
nalazi prirodan broj $B$.

Ako $i$-ti redak započinje riječju \texttt{cecilija}, tada se u nastavku retka
najprije nalazi prirodan broj $l_c$ (duljina niza $c$), a zatim se nalaze
elementi niza $c$ u neopadajućem poretku ($c_1 \le c_2 \le \ldots \le c_{l_c}$).

%%%%%%%%%%%%%%%%%%%%%%%%%%%%%%%%%%%%%%%%%%%%%%%%%%%%%%%%%%%%%%%%%%%%%%
% Output
\subsection*{Izlazni podaci}
U $i$-tom od $Q$ redaka treba ispisati odgovor na $i$-ti scenarij iz ulaza.

Ako je $i$-ti scenarij iz ulaza bio oblika \texttt{alojzije $A$}, tada treba
najprije ispisati broj $0 \le l_a \le L$ (duljinu niza $a$), a zatim elemente niza
$a$ koji predstavljaju niz brojeva koje bi Alojzije napisao na papir nakon što
mu gospodin Malnar pokaže broj $A$. Elementi niza $a$ ne smiju biti veći od
$10^9$.

Ako je $i$-ti scenarij iz ulaza bio oblika \texttt{benjamin $B$}, tada treba
najprije ispisati broj $0 \le l_b \le L$ (duljinu niza $b$), a zatim elemente niza $b$ koji
predstavljaju niz brojeva koje bi Benjamin napisao na papir nakon što mu gospodin
Malnar pokaže broj $B$. Elementi niza $b$ ne smiju biti veći od $10^9$.

Ako je $i$-ti scenarij iz ulaza bio oblika \texttt{cecilija $l_c$ $c_1$ ...
$c_{l_c}$}, tada treba ispisati \texttt{"A"} ako bi Cecilija na temelju niza
$c$ zaključila da je $A > B$, odnosno treba ispisati \texttt{"B"} bi Cecilija
zaključila da je $A < B$. Možete pretpostaviti da će niz $c$ biti generiran na
temelju nizova $a$ i $b$, čije su duljine bile najviše $L$, te  koje je vaš
program ispisao prilikom obrade scenarija \texttt{alojzije A} i
\texttt{benjamin B}. Pritom je moguće da je vaš program takve nizove $a$ i $b$
ispisao u nekom od prethodnih izvođenja.

%%%%%%%%%%%%%%%%%%%%%%%%%%%%%%%%%%%%%%%%%%%%%%%%%%%%%%%%%%%%%%%%%%%%%%
% Scoring
\subsection*{Bodovanje}
Vaše rješenje bit će testirano u dva koraka. Prvo će biti pozvano na testnom
podatku u kojem su svi scenariji oblika \texttt{alojzije A} ili
\texttt{benjamin B}. Uz pretpostavku da vaš program obradi sve scenarije u
ispravnom formatu, vaše će rješenje biti pozvano na testnom podatku u kojem
su svi scenariji započinju riječju \texttt{cecilija}, a pripadajući nizovi $c$
bit će generirani temeljem raznih kombinacija nizova $a$ i $b$ koje je vaš
program ispisao u prethodnom izvođenju. Vrijednost ulaznog podatka $L$ bit će
jednaka prilikom oba pokretanja. Ako vaš program u drugom izvođenju ispravno
odgovori na sve scenarije, vaše rješenje smatrat će se točnim.

Vrijeme izvršavanja vašeg rješenja je zbroj vremena izvršavanja
oba koraka evaluacije.

Označimo li s $N$ najveću vrijednost koju će u test podacima nekog podzadatka
poprimiti brojevi $A$ ili $B$, vaša će se rješenja bodovati prema sljedećoj
tablici:

{\renewcommand{\arraystretch}{1.4}
  \setlength{\tabcolsep}{6pt}
  \begin{tabular}{ccl}
   Podzadatak & Broj bodova & Ograničenja \\ \midrule
    1 & 11 & $N = 100$, $L = 200$, $1 \le Q \le 10\,000$ \\
    2 & 23 & $N = 1\,000$, $L = 110$, $1 \le Q \le 1\,000\,000$ \\
    3 & 66 & $N = 500\,000$, $L = 20$, $1 \le Q \le 1\,000\,000$
\end{tabular}}

%%%%%%%%%%%%%%%%%%%%%%%%%%%%%%%%%%%%%%%%%%%%%%%%%%%%%%%%%%%%%%%%%%%%%%
% Examples
\subsection*{Probni primjer}
\subsubsection*{Prvo pokretanje}
{\renewcommand{\arraystretch}{1.4}
  \setlength{\tabcolsep}{6pt}
  \begin{tabular}{ccl}
    Ulaz & Izlaz & Napomena \\ \midrule
     200 & & Nizovi koje Alojzije i Benjamin zapisuju smiju imati najviše $200$ elemenata. \\
     3   & & Potrebno je obraditi $3$ scenarija. \\
    \texttt{\frenchspacing alojzije 1} & \texttt{1 23} & Alojzije na temelju broja $1$ zapisuje niz $a = (23)$ \\
    \texttt{\frenchspacing benjamin 2} & \texttt{1 42} & Benjamin na temelju broja $2$ zapisuje niz $b = (42)$ \\
    \texttt{\frenchspacing alojzije 3} & \texttt{2 11 11} & Alojzije na temelju broja $3$ zapisuje niz $a = (11, 11)$ \\
\end{tabular}}

\subsubsection*{Drugo pokretanje}
{\renewcommand{\arraystretch}{1.4}
  \setlength{\tabcolsep}{6pt}
  \begin{tabular}{ccl}
    Ulaz & Izlaz & Napomena \\ \midrule
     200 & & Nizovi koje Alojzije i Benjamin zapisuju smiju imati najviše $200$ elemenata. \\
     2   & & Potrebno je obraditi $2$ scenarija. \\
    \texttt{\frenchspacing cecilija 2 1 1} & \texttt{B} & Niz $c = (1, 1)$ nastao je temeljem nizova $a = (23)$ i $b = (42)$, pa je $A < B$. \\
    \texttt{\frenchspacing cecilija 2 1 2} & \texttt{A} & Niz $c = (1, 2)$ nastao je temeljem nizova $a = (11, 11)$ i $b = (42)$, pa je $A > B$.\\
\end{tabular}}


%%%%%%%%%%%%%%%%%%%%%%%%%%%%%%%%%%%%%%%%%%%%%%%%%%%%%%%%%%%%%%%%%%%%%%
% We're done
\end{statement}

%%% Local Variables:
%%% mode: latex
%%% mode: flyspell
%%% ispell-local-dictionary: "croatian"
%%% TeX-master: "../hio.tex"
%%% End:
