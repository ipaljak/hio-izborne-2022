%%%%%%%%%%%%%%%%%%%%%%%%%%%%%%%%%%%%%%%%%%%%%%%%%%%%%%%%%%%%%%%%%%%%%%
% Problem statement
\begin{statement}[
  problempoints=100,
  timelimit=1 sekunda,
  memorylimit=512 MiB,
]{Mensza}

Gospodin Malnar nedavno je osnovao Menszu -- najveću, najprestižniju i jedinu
međunarodnu udrugu visokointeligentnih ljubitelja gastronomije. Pogađate, članom
udruge ne može postati bilo tko, već samo kandidati koji postignu izvrsne
rezultate na prijamnom ispitu. Dakako, ispit se sastoji od zahtjevnog IQ dijela
i vrlo zahtjevnog gurmanskog \sout{di}jela. Primjer tipičnog IQ dijela ispita
predstavit ćemo vam u ovom zadatku, dok ćete primjer gurmanskog jela imati
prilike okusiti tek nakon natjecanja.

Prijamnom ispitu pristupili su Alojzije, Benjamin i Cesarica. Gospodin Malnar
ih je odmjerio, brzo smislio pravedan zadatak, te se obratio Alojziju:

\textit{Alojzije, najprije ću tebe pozvati u svoj ured i pokazat ću ti prirodan
  broj $A$.  Zatim ćeš mi ti na papir napisati niz brojeva $a = (a_1, a_2,
\ldots, a_{l_a})$.}

Nakon što se uvjerio da je Alojzije shvatio svoj dio posla, obratio se Benjaminu:

\textit{Benjamine, tebe ću sljedećeg pozvati u ured i pokazat ću ti prirodan
  broj $B \ne A$.  Zatim ćeš mi ti na papir napisati niz brojeva $b = (b_1, b_2,
\ldots, b_{l_b})$.}

Potom se obratio i Cesarici:

\textit{Cesarice, tebe ću posljednju pozvati u ured i pokazat ću ti niz $c$ koji
  sam odredio na temelju nizova $a$ i $b$. Precznije, za svaki broj koji se barem
  jednom pojavljuje u nizovima $a$ i $b$, u niz $c$ ću dodati koliko se ukupno puta
  pojavljuje u uniji nizova $a$ i $b$, a elemente niza $c$ ću ti pokazati u
  neopadajućem poretku. Primjerice, ako je $a = (1, 2, 4)$ i $b = (1, 1, 2, 3)$,
  pokazat ću ti $c = (1, 1, 2, 3)$ zato što se brojevi $3$ i $4$ pojavljuju jednom,
  broj $2$ dva puta i broj $1$ tri puta. Nakon što ti pokažem niz $c$, ti mi trebaš
  odgovoriti koji je od brojeva $A$ i $B$ veći.}

Konačno, kada su se dojmovi malo slegli, obratio se i svim kandidatima zajedno:

\textit{Imate 60 minuta da smislite strategiju pa krećemo s ispitom. Nakon toga
  više ne smijete komunicirati. Ovaj postupak ćemo ponoviti par puta 
  \sout{da se uvjerim da se ne radi o sreći} dok ne stigne klopa.}

Vaš je zadatak osmisliti strategiju koja bi omogučila Alojziju, Benjaminu i
Cesarici da prođu IQ dio ispita.

%%%%%%%%%%%%%%%%%%%%%%%%%%%%%%%%%%%%%%%%%%%%%%%%%%%%%%%%%%%%%%%%%%%%%%
% Input
\subsection*{Ulazni podaci}
U prvom je retku prirodan broj $Q$, broj scenarija koje morate obraditi. Svaki
scenarij odgovara nekoj interakciji koja se zbiva u uredu gospodina Malnara.

U $i$-tom od sljedećih $Q$ redaka opisan je $i$-ti scenarij. Redak će započeti
riječju \texttt{alojzije}, \texttt{benjamin} ili \texttt{cesarica}, ovisno o
tome kojeg je kandidata gospodin Malnar pozvao u ured.

Ako $i$-ti redak započinje riječju \texttt{alojzije}, tada se u nastavku retka
nalazi prirodan broj $A$.

Ako $i$-ti redak započinje riječju \texttt{benjamin}, tada se u nastavku retka
nalazi prirodan broj $B$.

Ako $i$-ti redak započinje riječju \texttt{cesarica}, tada se u nastavku retka
najprije nalazi prirodan broj $l_c$ (duljina niza $c$), a zatim se nalaze
elementi niza $c$ u neopadajućem poretku ($c_1 \le c_2 \le \ldots \le c_{l_c}$).

%%%%%%%%%%%%%%%%%%%%%%%%%%%%%%%%%%%%%%%%%%%%%%%%%%%%%%%%%%%%%%%%%%%%%%
% Output
\subsection*{Izlazni podaci}
U $i$-tom od $Q$ redaka treba ispisati odgovor na $i$-ti scenarij iz ulaza.

Ako je $i$-ti scenarij iz ulaza bio oblika \texttt{alojzije $A$}, tada treba
najprije ispisati broj $l_a$ (duljinu niza $a$), a zatim elemente niza $a$ koji
predstavljaju niz brojeva koje bi Alojzije napisao na papir nakon što mu gospodin
Malnar pokaže broj $A$.

Ako je $i$-ti scenarij iz ulaza bio oblika \texttt{benjamin $B$}, tada treba
najprije ispisati broj $l_b$ (duljinu niza $b$), a zatim elemente niza $b$ koji
predstavljaju niz brojeva koje bi Benjamin napisao na papir nakon što mu gospodin
Malnar pokaže broj $B$.

Ako je $i$-ti scenarij iz ulaza bio oblika \texttt{cesarica $l_c$ $c_1$ ...
$c_{l_c}$}, tada treba ispisati \texttt{"A"} ako bi Cesarica na temelju niza
$c$ zaključila da je $A > B$, odnosno treba ispisati \texttt{"B"} bi Cesarica
zaključila da je $A < B$. Možete pretpostaviti da će niz $c$ biti generiran na
temelju nizova $a$ i $b$ koje je vaš program ispisao prilikom obrade scenarija
\texttt{alojzije A} i \texttt{benjamin B}. Pritom je moguće da je vaš program
takve nizove $a$ i $b$ ispisao u nekom od prethodnih izvođenja.

%%%%%%%%%%%%%%%%%%%%%%%%%%%%%%%%%%%%%%%%%%%%%%%%%%%%%%%%%%%%%%%%%%%%%%
% Scoring
\subsection*{Bodovanje}
Vaše rješenje bit će testirano u dva koraka. Prvo će biti pozvano na testnom
podatku u kojem su svi scenariji oblika \texttt{alojzije A} ili
\texttt{benjamin B}. Uz pretpostavku da vaš program obradi sve scenarije u
ispravnom formatu, vaše će rješenje biti pozvano na testnom podatku u kojem
su svi scenariji započinju riječju \texttt{cesarica}, a pripadajući nizovi $c$
bit će generirani temeljem raznih kombinacija nizova $a$ i $b$ koje je vaš
program ispisao u prethodnom izvođenju.

Ako vaš program u drugom izvođenju ispravno odgovori na sve scenarije, bodovat
će se prema


Prvo će biti pozvano sa
službenim ulaznim podatcima u kojima je $M=1$. Ako je izlaz vašeg
rješenja niz od $N$ valjanih matrica, u drugom koraku vaše rješenje bit
će ponovno pokrenuto s matricama ispisanima u prvom koraku.
Ako su stringovi koje vaš program ispiše u drugom koraku jednaki
stringovima u službenim ulaznim podatcima, dobit ćete sve bodove za
taj službeni ulaz.

Vrijeme izvršavanja vašeg rješenja je zbroj vremena izvršavanja
oba koraka evaluacije.

{\renewcommand{\arraystretch}{1.4}
  \setlength{\tabcolsep}{6pt}
  \begin{tabular}{ccl}
   Podzadatak & Broj bodova & Ograničenja \\ \midrule
    1 & 10 & $1 \le l_i \le 3$ \\
    2 & 10 & $1 \le l_i \le 7$ \\
    3 & 4 & $1 \le l_i \le 11$ i stringovi iz ulaza su nasumično generirani\\
    4 & 6 & $1 \le l_i \le 11$ \\
    5 & 4 & $1 \le l_i \le 12$ i stringovi iz ulaza su nasumično generirani\\
    6 & 6 & $1 \le l_i \le 12$ \\
    7 & 7 & $1 \le l_i \le 13$ i stringovi iz ulaza su nasumično generirani\\
    8 & 8 & $1 \le l_i \le 13$ \\
    9 & 9 & $1 \le l_i \le 14$ i stringovi iz ulaza su nasumično generirani\\
   10 & 11 & $1 \le l_i \le 14$ \\
   11 & 12 & $1 \le l_i \le 15$ i stringovi iz ulaza su nasumično generirani\\
   12 & 13 & $1 \le l_i \le 15$
\end{tabular}}

%%%%%%%%%%%%%%%%%%%%%%%%%%%%%%%%%%%%%%%%%%%%%%%%%%%%%%%%%%%%%%%%%%%%%%
% Examples
\subsection*{Probni primjer}
% \begin{tabularx}{\textwidth}{X'X}
% \sampleinputs{test/novine.dummy.in.1}{sample/a1out} &
% \sampleinputs{sample/a2in}{sample/a2out}
% \end{tabularx}


%%%%%%%%%%%%%%%%%%%%%%%%%%%%%%%%%%%%%%%%%%%%%%%%%%%%%%%%%%%%%%%%%%%%%%
% We're done
\end{statement}

%%% Local Variables:
%%% mode: latex
%%% mode: flyspell
%%% ispell-local-dictionary: "croatian"
%%% TeX-master: "../hio.tex"
%%% End:
