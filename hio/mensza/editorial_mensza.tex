\subsection*{Zadatak Mensza}
\textsf{Pripremio: Ivan Paljak }\\
\textsf{Potrebno znanje: prikaz prirodnog broja u binarnom zapisu}

Fokusirajmo se najprije na prva dva podzadatka kojima smo mogli pristupiti
na ``direktan'' način. Odnosno, moguće je osmisliti postupak kojim će Alojzije
i Benjamin u nizove $a$ i $b$ enkodirati tko su i koji su broj vidjeli u
uredu Gospodina Malnara.

U prvom podzadatku je postupak enkodiranja relativno jednostavan. Primjerice,
Alojzije je mogao svoj identitet i broj $A$ enkodirati nizom koji se sastoji od
$2A$ jedinica, a Benjamin je mogao svoj identitet i broj $B$ enkodirati nizom
koji se sastoji od $2B - 1$ dvojki.  Budući da Alojzije i Benjamin nikad neće
imati zajednički broj u svojim nizovima, te da im se nizovi sastoje od
uzastopnog ponavljanja jednog broja, zaključujemo da će se niz $c$ sastojati od
dva broja, a to su upravo $2A$ i $2B - 1$. Cecilija će tada parni element niza
$c$ podijeliti s $2$ kako bi dekodirala broj $A$, te će na sličan naćin
dekodirati i broj $B$, pa samim time jednostavno odrediti koji je veći.

Na sličan način ćemo pristupiti i drugom podzadatku, samo ćemo osmisliti nešto
efikasniji način enkodiranja brojeva $A$ i $B$. Poslužit ćemo se pritom zapisom
brojeva u binarnom sustavu, te ćemo ideju iz prethodnog odlomka primijeniti na
svaki bit zasebno. Preciznije, Alojzije će za $i$-ti postavljen bit u binarnom
zapisu broja $A$ u niz $a$ dodati broj $2i + 2$ točno $2i + 2$ puta. Slično,
Benjamin će za $j$-ti postavljen bit u binarnom zapisu broja $B$ u niz $b$ dodati
broj $2j + 1$ točno $2j + 1$ puta. Lako je primijetiti da će Cecilija na temelju
parnosti odrediti koji brojevi odgovaraju bitovima broja $A$, a koji brojevi
bitovima broja $B$.

Ograničenja za posljednji podzadatak su prestroga za dosadašnju strategiju.
Odnosno, morat ćemo odustati od ideje će Cecilija rekonstruirati brojeve $A$
i $B$.

Promatrat ćemo i dalje brojeve u binarnom zapisu i zapitati se što vrijedi kada
je $A > B$. Vrijedi da brojevi $A$ i $B$ imaju neki zajednički prefiks binarnih
znamenaka, a sljedeća znamenka je jedinica u broju $A$ i nula u broju $B$.
Primijetite da je pozicija tog bita jedinstvena, odnosno postoji samo jedna
pozicija takva da se na toj poziciji u broju $A$ nalazi jedinica, u broju $B$
nula, a prefiks do te pozicije je jednak u oba broja. Nazovimo taj bit
\textit{bitnim}. Promatrajmo sve pozicije gdje bi se mogao nalaziti bitni bit.
Naravno, u broju $A$ to su sve pozicije gdje se nalazi jedinica, a u broju $B$
to su sve pozicije gdje se nalaze nule. Ideja je na temelju broja $A$ generirati
skup svih prefiksa do pozicije gdje bi se mogao nalaziti bitni bit, te istu stvar
napraviti za broj $B$, samo što ćemo tamo invertirati bitni bit. Primijetite da,
ako je $A > B$, u nizu brojeva koje generiramo temeljem broja $A$ i nizu brojeva
koje genriramo temeljem broja $B$, postoji točno jedan zajednički element. Ako je
$A < B$, lako se uvjeriti da ne postoji zajednički element.

Dakle, Alojzije će najprije pretvoriti broj $A$ u binarni zapis, te u niz $a$
zapisati sve prefikse koji završavaju jedinicom, a sufiks popuniti nulama.
Benjamin će najprije pretvoriti broj $B$ u binarni zapis, te u niz $b$ zapisati
sve prefikse koji završavaju nulom, ali će tu nulu pretvoriti u jedinicu i ostatak
sufiksa popuniti nulama. Ceclija će zaključiti da je $A > B$ ako $a$ i $b$
imaju zajednički element.
