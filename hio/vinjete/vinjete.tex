%%%%%%%%%%%%%%%%%%%%%%%%%%%%%%%%%%%%%%%%%%%%%%%%%%%%%%%%%%%%%%%%%%%%%%
% Problem statement
\begin{statement}[
  problempoints=100,
  timelimit=1 sekunda,
  memorylimit=512 MiB,
]{Vinjete}

Nakon dvije godine \textit{on-line} izvedbe, međunarodna informatička
olimpijada (IOI) održat će se uživo. Znanstveno i tehničko povjerenstvo je
pod uobičajenim stresom, natjecatelji su uzbuđeni, roditelji nervozni, a
najviše od svih putovanju se raduje Gospodin Malnar. Ponovno će okusiti
ranojutarnji sok od grožđa na zagrebačkom aerodromu, ponovno će okusiti
najfinije delicije azijske gastro ponude, ponovno će uživati u dnevnim
izletima.

Iskusniji među vama zapitat će se: ``Kakvim izletima? Gospodin Malnar skoro
nikada ne sudjeluje na organiziranim ekskurzijama s ostalim delegacijama\ldots''.
U pravu ste, Gospodin Malnar ne sudjeluje na organiziranim ekskurzijama, već sam
planira svoje izlete mjesecima unaprijed.

Najprije riješi logistiku oko najma automobila, a zatim napravi kraći popis od
$N$ gradova koje bi volio posjetiti te ih istakne na karti kontinenta na
kojem se te godine održava natjecanje. Na kartu dodatno ucrta sve autoceste
koje spajaju neka dva istaknuta grada. Zanimljivo je da je ove godine na
kartu Azije ucrtao točno $(N-1)$-u autocestu, te da postoji put između svaka
dva grada koristeći ucrtane autceste.

I to nije sve, Gospodin Malnar je utvrdio da u Aziji postoji $M$ različitih
vrsta vinjeta, te da je za prelazak svake autoceste potrebno kupiti neki
podskup tih vinjeta. Gospodin Malnar odmah je numerirao različite vrste
vinjeta prirodnim brojevima od $1$ do $m$, a posebno je zanimljivo da ih je
uspio numerirati tako da je za prolazak $i$-te autoceste potrebno kupiti sve
vinjete koje imaju oznaku veću ili jednaku $l_i$ i manju ili jednaku $r_i$.

Gradove je pak numerirao prirodnim brojevima od $1$ do $N$ i to tako da je
Yogyakarta, grad u Indoneziji u kojem se održava natjecanje, označen brojem $1$.

Kako bi bolje isplanirao svoju rutu, odlučio vas je zamoliti da napišete program
koji će za svaki grad odrediti koliko najmanje vinjeta mora kupiti da bi
proputovao od Yogykarte do tog grada.
%%%%%%%%%%%%%%%%%%%%%%%%%%%%%%%%%%%%%%%%%%%%%%%%%%%%%%%%%%%%%%%%%%%%%%
% Input
\subsection*{Ulazni podaci}

U prvom su retku prirodni brojevi $N$ i $M$ iz teksta zadatka.

U $i$-tom od sljedećih $N-1$ nalazi se četvorka $a_i$, $b_i$, $l_i$ i $r_i$
koja nam govori da $i$-ta autocesta spaja gradove s oznakama $a_i$ i $b_i$
$(1 \le a_i, b_i \le N, a_i \ne b_i)$, te da je za njen prolazak potrebno
kupiti vinjete s oznakama iz intervala $[l_i, r_i]$ $(1 \le l_i \le r_i \le
M)$.

Autoceste će biti takve da povezuju svaki par od istaknutih $N$ gradova.

%%%%%%%%%%%%%%%%%%%%%%%%%%%%%%%%%%%%%%%%%%%%%%%%%%%%%%%%%%%%%%%%%%%%%%
% Output
\subsection*{Izlazni podaci}

U $i$-ti od $N - 1$ redaka ispišite najmanji broj vinjeta koje Gospodin Malnar
kupiti da bi proputovao od Yogyakarte (grada s oznakom $1$) do grada s
oznakom $i+1$.

%%%%%%%%%%%%%%%%%%%%%%%%%%%%%%%%%%%%%%%%%%%%%%%%%%%%%%%%%%%%%%%%%%%%%%
% Scoring
\subsection*{Bodovanje}

U svim podzadacima vrijedi $1 \leq N \leq 500~000$ i $0 \leq M \leq 500~000$.

{\renewcommand{\arraystretch}{1.4}
  \setlength{\tabcolsep}{6pt}
  \begin{tabular}{ccl}
   Podzadatak & Broj bodova & Ograničenja \\ \midrule
    1 & ?? & ?? \\
    2 & ?? & ?? \\
    3 & ?? & ?? \\
    4 & ?? & ??
\end{tabular}}

%%%%%%%%%%%%%%%%%%%%%%%%%%%%%%%%%%%%%%%%%%%%%%%%%%%%%%%%%%%%%%%%%%%%%%
% Examples
\subsection*{Probni primjer}
\begin{tabularx}{\textwidth}{X'X'X}
\sampleinputs{test/vinjete.dummy.in.1}{test/vinjete.dummy.out.1} &
\sampleinputs{test/vinjete.dummy.in.2}{test/vinjete.dummy.out.2} &
\sampleinputs{test/vinjete.dummy.in.3}{test/vinjete.dummy.out.3}
\end{tabularx}

\textbf{Pojašnjenje probnih primjera:}

??

%%%%%%%%%%%%%%%%%%%%%%%%%%%%%%%%%%%%%%%%%%%%%%%%%%%%%%%%%%%%%%%%%%%%%%
% We're done
\end{statement}

%%% Local Variables:
%%% mode: latex
%%% mode: flyspell
%%% ispell-local-dictionary: "croatian"
%%% TeX-master: "../hio.tex"
%%% End:
