%%%%%%%%%%%%%%%%%%%%%%%%%%%%%%%%%%%%%%%%%%%%%%%%%%%%%%%%%%%%%%%%%%%%%%
% Problem statement
\begin{statement}[
  problempoints=100,
  timelimit=3 seconds,
  memorylimit=512 MiB,
]{Povjerenstvo}

Do you know how hard it is to choose a set of people for the problem selection 
committee? No? Well do you know who does? Mr.\ Malnar, of course. By observing 
human interactions, the all-knowing Mr.\ Malnar has decided what the ideal 
choice should look like.

A total of $N$ people are being considered for the committee and $M$ relations 
between them have been recorded. A relation is described by an ordered pair 
$(a, b)$ representing the fact that person $a$ dislikes person $b$.  Mr.\ Malnar 
defines a \textit{dislike circle} to be a sequence of distinct people $x_1, x_2, \dots, x_k$
such that person $x_i$ dislikes person $x_{i+1}$, for each $1 \leq i \leq k$ 
(it is assumed that $x_{k+1} = x_1$). Mr.\ Malnar noticed a peculiar property 
regarding the set of people in question: \textbf{there is no dislike circle consisting 
of an odd number of people}.

To minimize dissatisfaction with the choice of committee, Mr.\ Malnar is looking for a 
committee such that everyone within the committee agrees with each other and everyone 
outside of the committee is glad not to be in it. More precisely:
\begin{itemize}
    \item There must not be two people within the committee such that one person dislikes the other.
    \item For each person outside the committee there should be someone in the committee who they dislike.
\end{itemize}

Can you find such a set of people?

%%%%%%%%%%%%%%%%%%%%%%%%%%%%%%%%%%%%%%%%%%%%%%%%%%%%%%%%%%%%%%%%%%%%%%
% Input
\subsection*{Ulazni podaci}

The first line contains positive integers $N$ and $M$, the number of people and number
of relations between them, respectively.

The $i$-th of the following $M$ lines contains an ordered pair of positive 
integers $a_i$ and $b_i$ ($1 \leq a_i, b_i \leq N$), representing the fact 
that person $a_i$ dislikes the person $b_i$. It holds that $a_i \ne b_i$ for 
all $i = 1, 2, \dots M$ and no ordered pair is listed twice.

The given relations will be such that there is no dislike circle consisting 
of an odd number of people.

%%%%%%%%%%%%%%%%%%%%%%%%%%%%%%%%%%%%%%%%%%%%%%%%%%%%%%%%%%%%%%%%%%%%%%
% Output
\subsection*{Izlazni podaci}

If it is not possible to choose a set of people satisfying the given conditions, 
in the only line print \texttt{-1}.

Otherwise, in the first line print a positive integer $K$ ($1 \leq K \leq N$), 
the number of people in the committee. In the next line print $K$ distinct 
positive integers $p_1, p_2, \dots p_K$ ($1 \leq p_i \leq N$), the indices of 
the people which make up the committee.

If there is more than one solution, output any one of them.

%%%%%%%%%%%%%%%%%%%%%%%%%%%%%%%%%%%%%%%%%%%%%%%%%%%%%%%%%%%%%%%%%%%%%%
% Scoring
\subsection*{Bodovanje}

In all subtasks, it holds that $1 \leq N \leq 500~000$ and $0 \leq M \leq 500~000$.

{\renewcommand{\arraystretch}{1.4}
  \setlength{\tabcolsep}{6pt}
  \begin{tabular}{ccl}
   Subtask & Score & Constraints \\ \midrule
    1 & 13 & There is no dislike circle. \\
    2 & 21 & There is no sequence of people of odd length $x_1, x_2, \dots x_k$ such that \\
      &    & one of $x_i$ or $x_{i+1}$ dislikes the other, for all $1 \leq i \leq k$. \\
    3 & 33 & $N, M \leq 5000$ \\
    4 & 33 & No additional constraints.
\end{tabular}}

%%%%%%%%%%%%%%%%%%%%%%%%%%%%%%%%%%%%%%%%%%%%%%%%%%%%%%%%%%%%%%%%%%%%%%
% Examples
\subsection*{Probni primjeri}
\begin{tabularx}{\textwidth}{X'X'X}
\sampleinputs{test/povjerenstvo.dummy.in.1}{test/povjerenstvo.dummy.out.1} &
\sampleinputs{test/povjerenstvo.dummy.in.2}{test/povjerenstvo.dummy.out.2} &
\sampleinputs{test/povjerenstvo.dummy.in.3}{test/povjerenstvo.dummy.out.3}
\end{tabularx}

\textbf{Explanation of the examples:}

The set of chosen people is shown in the output of each test case.

The first example is a valid test case for the first subtask and for the second subtask.

The second example is not a valid test case for the first subtask, but it is valid for the second subtask.

The third example is not a valid test case for the first subtask nor for the second subtask.

%%%%%%%%%%%%%%%%%%%%%%%%%%%%%%%%%%%%%%%%%%%%%%%%%%%%%%%%%%%%%%%%%%%%%%
% We're done
\end{statement}

%%% Local Variables:
%%% mode: latex
%%% mode: flyspell
%%% ispell-local-dictionary: "croatian"
%%% TeX-master: "../hio.tex"
%%% End:
