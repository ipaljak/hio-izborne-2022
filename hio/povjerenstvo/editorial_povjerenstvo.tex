\subsection*{Zadatak Povjerenstvo}
\textsf{Pripremio: Krešimir Nežmah}\\
\textsf{Potrebno znanje: topološko sortiranje, strongly-connected komponente}

Jasno se nameće interpretacija situacije iz zadatka kao usmjereni graf. Svaka od $N$ osoba predstavljena je čvorom, a ako osoba $b$ ide na živce osobi $a$, to ćemo predstaviti usmjerenim bridom iz $a$ u $b$. Uvjet iz zadatka nam tada govori da u danom grafu ne postoji usmjereni ciklus neparne duljine, a potrebno je pronaći skup čvorova tako da nikoja dva čvora iz skupa nisu susjedna te da svaki čvor izvan skupa pokazuje u neki unutar skupa. Ispostavlja se da rješenje uvijek postoji. Opisat ćemo kako konstruirati traženi skup za svaki podzadatak.

U prvom podzadatku dani graf nema usmjereni ciklus, dakle riječ je o DAG-u. Promatrajmo čvorove kojima je out-degree jednak 0, tj. one čvorove koji nemaju izlaznih bridova. Nijedan takav čvor ne može biti izvan povjerenstva jer onda ne bi mogao pokazivati u neki čvor unutar povjerenstva. Prema tome, svi takvi čvorovi nužno moraju pripadati povjerenstvu, a nakon što odaberemo te čvorove, znamo da ne smijemo odabrati čvorove koji u njih pokazuju. Čvorove koje u njih pokazuju zato možemo izbrisati iz grafa te na preostalom grafu ponavljati postupak. Spomenutu ideju možemo implementirati u složenosti $O(N + M)$ pomoću \textit{queuea} u kojeg ubacujemo čvorove kada im out-degree postane 0.

U drugom podzadatku dani graf nema neparni ciklus pa je bipartitan. Slično kao i u rješenju za prvi podzadatak, možemo micati čvorove dok god postoji barem jedan čvor čiji je out-degree jednak 0. Nakon toga preostaje graf u kojem svaki čvor pokazuje u barem jedan drugi čvor i tada je dovoljno obojati čvorove u dvije boje (što je po pretpostavci moguće) te izabrati sve čvorove jedne boje kao dio povjerenstva. Zbog bipartitnosti se neće dogoditi da postoji veza unutar povjerenstva, a svaki čvor izvan povjerenstva će pokazivati u čvorove u povjerenstvu te će ih po pretpostavci biti barem jedan.

Graf u kojem je moguće od svakog čvora doći do svakog drugog nazivamo \textit{strongly-connected}. Svaki je graf moguće particionirati na strongly-connected komponente, a ako svaku komponentu kompresiramo u jedan čvor dobivamo DAG. Dekompozicija se može provesti koristeći poznate algoritme za rastav na strongly-connected komponente, poput Kosaraju-ovog algoritma. Ključna opservacija za treći i četvrti podzadatak je da graf nema usmjereni ciklus neparne duljine ako i samo ako je svaka strongly-connected komponenta bipartitna (gledajući bridove kao neusmjerene). U sljedećem je paragrafu dokaz spomenute tvrdnje.

Ako graf ima neparan usmjereni ciklus, taj ciklus je onda u sadržan unutar jedne strongly-connected komponente. Tada ta komponenta sadrži neparni ciklus čak i ako gledamo bridove neusmjereno pa ona nije bipartitna. Za drugi smjer, pretpostavimo da promatramo strongly-connected komponentu u kojoj je svaki usmjereni ciklus parne duljine. Želimo pokazati da je ta strongly-connected komponenta bipartitna. Pretpostavimo suprotno, da u njoj postoji ciklus neparne duljine te pokažimo da onda postoji i usmjereni ciklus neparne duljine. Svaki brid $u \leftarrow v$ koji je u promatranom neparnom ciklusu okrenut u krivu stranu možemo zamijeniti putom iz $u$ prema $v$ (koji sigurno postoji jer promatramo strongly-connected komponentu). Duljina tog puta je neparna jer zajedno s bridom $u \leftarrow v$ taj put mora tvoriti ciklus parne duljine. Stoga, kada zamijenimo brid s putom, duljina promatranog ciklusa je i dalje neparna. Postupnim zamjenama krivo usmjerenih bridova dolazimo do željene kontradikcije.

Treći podzadatak sada možemo riješiti na sljedeći način. Rastavimo graf na strongly-connected komponente te promatramo bilo koju komponentu čiji je out-degree prema ostalim komponentama jednak 0. Pobojamo tu komponentu u dvije boje te sve čvorove jedne boje uzmemo kao dio povjerenstva, a sve čvorove koje u njih pokazuju izbrišemo. Budući da brisanje čvorova može promijeniti rastav na komponente, nakon svakog ovakvog koraka ponovno radimo dekompoziciju. Ukupna složenost ovog pristupa je $O(N(N + M))$.

Četvrti podzadatak potrebno je riješiti bez da svaki put ponovno radimo dekompoziciju. Rastav na strongly-connected komponente napravit ćemo jednom na početku te iterirati po komponentama u obrnutom poretku topološkog sortiranja. Kada obrađujemo neku komponentu, moguće je da su neki čvorovi iz nje već izbrisani, no i nakon brisanja promatrana komponenta je bipartitna pa možemo naprosto iskoristiti rješenje drugog podzadatka. Dakle, kada smo na trenutnoj komponenti, brišemo čvorove dok god postoji jedan s out-degreeom jednakim 0, a nakon toga uzimamo bipartitno bojanje. Ukupna složenost je $O(N + M)$.
